\documentclass[a4paper,11pt]{exam}
\usepackage{amsmath} % Advanced math typesetting
\usepackage[T1]{fontenc}
\usepackage[utf8]{inputenc} % Unicode support (Umlauts etc.)
\usepackage{lmodern}
\usepackage[french]{babel}  % Change hyphenation rules
\usepackage{authblk}
\usepackage{ragged2e}
\usepackage{hyperref}
\usepackage{listings}
\usepackage{amssymb}
\usepackage{algorithm,algpseudocode}

\begin{document}

\title{\large TD1 : MPFR}
\author{Christophe Pont}
\affil{LAMPS, UPVD, 52 Av. Paul Alduy, 66860 Perpignan, France \newline
christophe.pont@univ-perp.fr
https://github.com/christof337/AlgoEtCalculScientifique}
\date{\today{}} % You can remove \today{} and type a date manually

\lstset{
	language=C, 
	breakatwhitespace=false,
	breaklines=true, 
	showspaces=false
	showstringspaces=false,          % underline spaces within strings only
  showtabs=false                  % show tabs within strings adding particular underscores
} 

%\maketitle{} % Generates title
\centering
{\huge\bfseries TD1 : MPFR\par}
\begin{flushleft}
{Christophe Pont, LAMPS \hfill \emph{Licence L3}\\ 
christophe.pont@univ-perp.fr \hfill \emph{Algo et Calcul scientifique} \\ 
https://github.com/christof337/AlgoEtCalculScientifique \hfill \today} \\ 
\end{flushleft}

\justifying
\section{Présentation de MPFR}
\paragraph{Présentation}
MPFR est une bibliothèque C utilisée pour des calculs en virgule flottante à précision arbitraire\footnote{\url{http://www.mpfr.org/\#intro}} - elle a été développée en France\footnote{\url{http://www.mpfr.org/credit.html}}. 
Elle est entièrement libre, et sous \emph{GNU Lesser GPL}, une licence sensiblement permissive\footnote{\url{http://www.gnu.org/copyleft/lesser.html}}.\\
Initialement intégrée à GMP\footnote{\url{https://gmplib.org/}}, elle en est désormais une extension. De facto, GMP est toujours nécessaire à l'utilisation de MPFR. \\
Bien que la bibliothèque ait été initialement développée en C, des interfaces existent pour d'autres langages\footnote{\url{http://www.mpfr.org/\#interfaces}} (attention, la conception est parfois différente) :
	\begin{itemize}
		\item \href{http://www.mpfr.org/\#interfaces}{C++}
		\item \href{https://github.com/kframework/mpfr-java}{Java}
		\item \href{http://search.cpan.org/~sisyphus/Math-MPFR/}{Perl}
		\item \href{https://pypi.python.org/pypi/gmpy2}{Python}
		\item \href{http://rubygems.org/gems/gmp}{Ruby}
		\item \href{https://cran.r-project.org/web/packages/Rmpfr/index.html}{R}
		\item ...
	\end{itemize}
\paragraph{Objectif} L'intérêt ici est de se familiariser avec cette bibliothèque, et l'utiliser afin de manipuler facilement la précision et les modes d'arrondis permis, en langage C.  
\paragraph{Documentation} Une documentation de l'interface est disponible en ligne (\url{http://www.mpfr.org/mpfr-current/mpfr.html}). \textbf{Elle servira de référence tout au long de ce TP}.

\section{Installation}
Si MPFR n'est pas présent sur votre système, vous pouvez le télécharger \href{http://www.mpfr.org/}{ici} ou \href{https://ftp.gnu.org/gnu/mpfr/}{là}.
L'OS recommandé pour l'utilisation de MPFR est bien évidemment GNU/Linux, de par la forte dépendance entre GMP et MPFR. Ceci étant, vous restez libres.
\paragraph{Sous Linux}
Bien que certaines distributions GNU/Linux l'intègrent nativement, les headers ne sont pas toujours disponibles aux emplacements adéquats au développement.\\
Se référer à \url{http://www.mpfr.org/mpfr-current/mpfr.html\#Installing-MPFR} pour les instructions d'installation détaillées, et pour tester si l'installation fonctionne.

\paragraph{Sous Windows}
Vous trouverez tout le nécessaire pour build les packages (ou en télécharger des \emph{prebuilt}) \href{https://github.com/emphasis87/libmpfr-msys2-mingw64}{ici} \url{https://github.com/emphasis87/libmpfr-msys2-mingw64}.

\paragraph{Sous OSX} 
Voir \href{https://github.com/macports/macports-ports}{macports} et récupérer le \emph{Portfile} de MPFR afin de procéder à l'installation (\url{https://github.com/macports/macports-ports/blob/master/devel/mpfr/Portfile}).

\paragraph{Tester l'installation}
Si la compilation du simple programme C suivant
\begin{lstlisting}
	#include <stdio.h>
	#include <mpfr.h>
	int main (void)
	{
		printf ("MPFR library: %-12s\nMPFR header: %s (based on %d.%d.%d)\n", mpfr_get_version(), MPFR_VERSION_STRING, MPFR_VERSION_MAJOR, MPFR_VERSION_MINOR, MPFR_VERSION_PATCHLEVEL);
		return 0;
	}
\end{lstlisting}
avec \verb=cc -o version version.c -lmpfr -lgmp= s'exécute correctement (et que la version s'affiche bien à l'exécution de \verb=./version=), alors MPFR est correctement installé : vous pouvez passer à la suite.

\section{Exercices}
Une lecture préliminaire pour comprendre la bibliothèque se trouve \href{http://www.mpfr.org/mpfr-current/mpfr.html#MPFR-Basics}{ici} \url{http://www.mpfr.org/mpfr-current/mpfr.html#MPFR-Basics}.\\
L'interface de référence quant à elle est \href{http://www.mpfr.org/mpfr-current/mpfr.html#MPFR-Interface}{là} \url{http://www.mpfr.org/mpfr-current/mpfr.html#MPFR-Interface}, habituez-vous à y naviguer.

\setcounter{enumi}{1}
\subsection{Exercice \theenumi : Factorielle}
\subsubsection{C types}
\emph{Conseil : vous pouvez trouver le }\verb=specifier= \emph{de chaque type ici \url{https://en.wikipedia.org/wiki/C_data_types}. Pratique pour les} \verb=printf=.
\setcounter{enumii}{1}
\paragraph{\theenumii.} 
Créer un programme C permettant d'afficher la factorielle des $n$ premiers entiers en utilisant le format \verb=int=. Utiliser si possible une fonction récursive. 
\stepcounter{enumii}
\paragraph{\theenumii.} 
Observer le comportement aux alentours des entiers $[12, 14]$. \\
\emph{Vous pouvez tester les valeurs des factorielles \href{https://fr.numberempire.com/factorialcalculator.php}{ici} si vous avez un doute}
\stepcounter{enumii}
\paragraph{\theenumii.} 
Réécrire la fonction pour un \verb=long int=, un \verb=float=. Afficher ensuite les factorielles des 36 premiers entiers et concluez.
\stepcounter{enumii}
\paragraph{\theenumii.} 
Recommencez avec un type \verb=double=. Tester pour $n \in \left[ 170, 172 \right]$

\subsubsection{MPFR}
\setcounter{enumii}{1}
Nous allons observer le comportement du même procédé sur des variables \verb=MPFR=.
\paragraph{\theenumii.} 
Consulter la documentation pour vous familiariser avec l'initialisation des variables \verb=mpfr_t=. \\
Trouver \href{http://www.mpfr.org/mpfr-current/mpfr.html#index-mpfr_005ffac_005fui}{une méthode} de la bibliothèque MPFR permettant de calculer la factorielle directement.
\stepcounter{enumii}
\paragraph{\theenumii.} 
Initialiser une variable \verb=mpfr_t= à l'aide de la méthode \href{http://www.mpfr.org/mpfr-current/mpfr.html#index-mpfr_005finit}{mpfr\_init}.\\
Appeler la méthode de la factorielle sur cette variable (en utilisant comme mode d'arrondi \verb=MPFR_RNDN=). \\
Comparer les résultats obtenus avec ceux d'un \verb=double= pour $n \in \left[ 170, 172 \right]$.\\ \\
Pour l'affichage, utiliser
\begin{lstlisting}
mpfr_out_str(stdout, 10, 0, mpfr_val_to_print, MPFR_RNDN);
\end{lstlisting} où \verb=mpfr_val_to_print= est la variable \verb=mpfr_t= à afficher.
\stepcounter{enumii}
\paragraph{\theenumii.} 
À l'aide de la méthode \href{http://www.mpfr.org/mpfr-current/mpfr.html#index-mpfr_005finit2}{mpfr\_init2}, choisir arbitrairement la précision de la mantisse.\\
Afficher les factorielles des 300 premiers entiers pour une mantisse de 200 bits.
\stepcounter{enumii}
\paragraph{\theenumii.} 
Concluez sur la flexibilité de la taille de l'exposant avec MPFR.
%\emph{On observera vite qu'il n'est pas possible de renvoyer un type mpfr\_t en retour d'une fonction}


\stepcounter{enumi}
\setcounter{enumii}{1}
\subsection{Exercice \theenumi : Modes d'arrondi}
Vous trouverez ici : \url{http://www.mpfr.org/sample.html} un exemple pratique d'utilisation de mpfr.
\paragraph{\theenumii.}En utilisant cet algorithme, sommer les inverses des factorielles des $n$ premiers entiers à l'aide de MPFR.
\stepcounter{enumii} 
\paragraph{\theenumii.}(Facultatif : MPFR gère les modes d'arrondis suivants : \url{http://www.mpfr.org/mpfr-current/mpfr.html#Rounding-Modes}. Changer le mode d'arrondi utilisé dans les calculs et observer l'impact sur les résultats. \\ 
Cet ordonnancement était-il prévisible?)

\stepcounter{enumi}
\setcounter{enumii}{1}

\subsection{Exercice \theenumi : Évaluation de polynome}
\subsubsection{Méthode directe}
Soit le polynôme $p(x) = x^{15} - 30x^{14} + 420x^{13} - 3640x^{12} + 21840x^{11} -96096x^{10} + 320320x^{9} -823680x^{8}+ 1647360x^{7} -2562560x^{6}+ 3075072x^{5} -2795520x^{4}+ 1863680x^{3} -860160x^{2}+ 245760x -32768 $.
\paragraph{\theenumii.} Créer un tableau correspondant à ce polynome à l'aide du code ci-dessous.
\begin{lstlisting}
	const double polynome[16] = { 1, -30, 420, -3640, 21840, -96096, 320320, -823680, 1647360, -2562560, 3075072, -2795520, 1863680, -860160, 245760, -32768 };
\end{lstlisting}

\stepcounter{enumii}
\paragraph{\theenumii.} Créer une méthode permettant d'évaluer ce polynome directement, pour une valeur $x$ donnée. Utiliser \verb=mpfr_t= pour les calculs avec l'arrondi \verb=MPFR_RNDN=.

\stepcounter{enumii}
\paragraph{\theenumii.} Evaluez ainsi $p(x)$ pour $x \in \left[ 1.6,2.4 \right]$ avec un pas de $10^{-4}$. \\
Afficher les résultats dans un fichier (première colonne : valeur de x, seconde colonne : valeur de son évaluation) ; idéalement avec une séparation par tabulation. Ploter ensuite le fichier pour observer le comportement de l'évaluation aux alentours de 2, avec Gnuplot\footnote{\url{http://gnuplot.sourceforge.net/}} par exemple.\\
Les oscillations observées rendent difficile la détection de la racine de $p$ par une méthode comme celle de la dichotomie par exemple.\\
\\A l'aide du graphique, prospectez la valeur d'une des racines du polynôme.

\subsubsection{Méthode de Horner}
\setcounter{enumii}{1}
Horner propose un algorithme plus efficace pour évaluer un polynome, évitant de recalculer les puissances de $x$ successives.\\
Le principe est de factoriser les $x$ un à un. \\
Soit \[ P\left(X\right) = a_{n}X^{n}+...+a_{1}X +  a_{0} \in  \mathbb{R}\left[ X\right] \]
La réduction de Horner donne
\[a_{n}x^{n}+...+a_{1}x + a_{0} = ((...(a_{n}x+a_{n-1})x+a_{n-2})...)x+a_{0} \]
\\ 
L'algorithme d'évaluation devient alors :\\
\begin{algorithmic}
 	\State $y\gets a_{n}$
 \For{$i\gets (n-1), 0$}
 		\State $y\gets x.y+a_{i}$
 \EndFor
\end{algorithmic}

Soit en C :
\begin{lstlisting}
	void evaluatePolynomeHorner(mpfr_t acc, const double xDouble) {
	mpfr_t x;

	mpfr_init2(x, PRECISION);

	mpfr_set_d(x, xDouble, MPFR_RNDN);
	mpfr_set_d(acc, polynome[0], MPFR_RNDN);

	for (size_t ind = 1; ind <= DEGRE; ++ind) {
//		acc = x * acc + polynome[ind];
		mpfr_mul(acc, acc, x, MPFR_RNDN);
		mpfr_add_d(acc, acc, polynome[ind], MPFR_RNDN);
	}

	mpfr_clear(x);
}
\end{lstlisting}

\paragraph{\theenumii.} Implémenter l'algorithme de Horner pour l'évaluation du polynome. L'appliquer au même intervalle que précédemment. Observer la différence.
\setcounter{enumii}{1}
\subsubsection{Factorisation}
En réalité, le polynome étudié peut-être factorisé en $(x-2)^{15}$.
\paragraph{\theenumii.}Utiliser cette dernière formule pour calculer directement les évaluations sur le même intervalle que précédemment.
\stepcounter{enumii}
\paragraph{\theenumii.}Concluez sur l'efficacité des méthodes (et l'intérêt de trouver les racines lorsque possible) en terme de temps de calcul et d'exactitude.

\setcounter{enumii}{1}
\subsubsection{MPFR}
\paragraph{\theenumii.}Utiliser \verb=mpfr_init2= (au lieu de \verb=mpfr_init=) pour augmenter la précision jusqu'à 60 (au lieu de 53).
\stepcounter{enumii}
\paragraph{\theenumii.}Observer l'évolution des oscillations, même avec une méthode directe. 
\stepcounter{enumii}
\paragraph{\theenumii.}Changer la précision à 8000. Observer l'évolution sur le temps de calcul. \\
Sur un procédé itératif, on aurait une déperdition conséquente en terme de temps de calcul... et de mémoire!


\end{document}
